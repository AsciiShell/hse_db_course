%!TEX TS-program = xelatex

% Шаблон документа LaTeX создан в 2018 году
% Алексеем Подчезерцевым
% В качестве исходных использованы шаблоны
% 	Данилом Фёдоровых (danil@fedorovykh.ru) 
%		https://www.writelatex.com/coursera/latex/5.2.2
%	LaTeX-шаблон для русской кандидатской диссертации и её автореферата.
%		https://github.com/AndreyAkinshin/Russian-Phd-LaTeX-Dissertation-Template

\documentclass[a4paper,14pt]{article}


%%% Работа с русским языком
\usepackage[english,russian]{babel}   %% загружает пакет многоязыковой вёрстки
\usepackage{fontspec}      %% подготавливает загрузку шрифтов Open Type, True Type и др.
\defaultfontfeatures{Ligatures={TeX},Renderer=Basic}  %% свойства шрифтов по умолчанию
\setmainfont[Ligatures={TeX,Historic}]{Times New Roman} %% задаёт основной шрифт документа
\setsansfont{Comic Sans MS}                    %% задаёт шрифт без засечек
\setmonofont{Courier New}
\usepackage{indentfirst}
\frenchspacing

\renewcommand{\epsilon}{\ensuremath{\varepsilon}}
\renewcommand{\phi}{\ensuremath{\varphi}}
\renewcommand{\kappa}{\ensuremath{\varkappa}}
\renewcommand{\le}{\ensuremath{\leqslant}}
\renewcommand{\leq}{\ensuremath{\leqslant}}
\renewcommand{\ge}{\ensuremath{\geqslant}}
\renewcommand{\geq}{\ensuremath{\geqslant}}
\renewcommand{\emptyset}{\varnothing}

%%% Дополнительная работа с математикой
\usepackage{amsmath,amsfonts,amssymb,amsthm,mathtools} % AMS
\usepackage{icomma} % "Умная" запятая: $0,2$ --- число, $0, 2$ --- перечисление

%% Номера формул
%\mathtoolsset{showonlyrefs=true} % Показывать номера только у тех формул, на которые есть \eqref{} в тексте.
%\usepackage{leqno} % Нумерация формул слева	

%% Перенос знаков в формулах (по Львовскому)
\newcommand*{\hm}[1]{#1\nobreak\discretionary{}
	{\hbox{$\mathsurround=0pt #1$}}{}}

%%% Работа с картинками
\usepackage{graphicx}  % Для вставки рисунков
\graphicspath{{images/}}  % папки с картинками
\setlength\fboxsep{3pt} % Отступ рамки \fbox{} от рисунка
\setlength\fboxrule{1pt} % Толщина линий рамки \fbox{}
\usepackage{wrapfig} % Обтекание рисунков текстом

%%% Работа с таблицами
\usepackage{array,tabularx,tabulary,booktabs} % Дополнительная работа с таблицами
\usepackage{longtable}  % Длинные таблицы
\usepackage{multirow} % Слияние строк в таблице
\usepackage{float}% http://ctan.org/pkg/float

%%% Программирование
\usepackage{etoolbox} % логические операторы


%%% Страница
\usepackage{extsizes} % Возможность сделать 14-й шрифт
\usepackage{geometry} % Простой способ задавать поля
\geometry{top=20mm}
\geometry{bottom=20mm}
\geometry{left=20mm}
\geometry{right=10mm}
%
%\usepackage{fancyhdr} % Колонтитулы
% 	\pagestyle{fancy}
%\renewcommand{\headrulewidth}{0pt}  % Толщина линейки, отчеркивающей верхний колонтитул
% 	\lfoot{Нижний левый}
% 	\rfoot{Нижний правый}
% 	\rhead{Верхний правый}
% 	\chead{Верхний в центре}
% 	\lhead{Верхний левый}
%	\cfoot{Нижний в центре} % По умолчанию здесь номер страницы

\usepackage{setspace} % Интерлиньяж
\onehalfspacing % Интерлиньяж 1.5
%\doublespacing % Интерлиньяж 2
%\singlespacing % Интерлиньяж 1

\usepackage{lastpage} % Узнать, сколько всего страниц в документе.

\usepackage{soul} % Модификаторы начертания

\usepackage{hyperref}
\usepackage[usenames,dvipsnames,svgnames,table,rgb]{xcolor}
\hypersetup{				% Гиперссылки
	unicode=true,           % русские буквы в раздела PDF
	pdftitle={Практическая по БД},   % Заголовок
	pdfauthor={Подчезерцев Алексей},      % Автор
	pdfsubject={Создание и заполнение отношений БД фитнес-клуба},      % Тема
	pdfcreator={Подчезерцев Алексей}, % Создатель
	pdfproducer={Подчезерцев Алексей}, % Производитель
	pdfkeywords={БД} {SQL} {MySQL}, % Ключевые слова
	colorlinks=true,       	% false: ссылки в рамках; true: цветные ссылки
	linkcolor=black,          % внутренние ссылки
	citecolor=black,        % на библиографию
	filecolor=magenta,      % на файлы
	urlcolor=black           % на URL
}
\makeatletter 
\def\@biblabel#1{#1. } 
\makeatother
\usepackage{cite} % Работа с библиографией
%\usepackage[superscript]{cite} % Ссылки в верхних индексах
%\usepackage[nocompress]{cite} % 
\usepackage{csquotes} % Еще инструменты для ссылок

\usepackage{multicol} % Несколько колонок

\usepackage{tikz} % Работа с графикой
\usepackage{pgfplots}
\usepackage{pgfplotstable}

% ГОСТ заголовки
\usepackage[font=small]{caption}
%\captionsetup[table]{justification=centering, labelsep = newline} % Таблицы по правобу краю
%\captionsetup[figure]{justification=centering} % Картинки по центру


\newcommand{\tablecaption}[1]{\addtocounter{table}{1}\small \begin{flushright}\tablename \ \thetable\end{flushright}%	
\begin{center}#1\end{center}}

\newcommand{\imref}[1]{Рис.~\ref{#1}}

\usepackage{multirow}
\usepackage{spreadtab}
\newcolumntype{K}[1]{@{}>{\centering\arraybackslash}p{#1cm}@{}}


\usepackage{xparse}
\ExplSyntaxOn
\DeclareExpandableDocumentCommand{\juliandate}{ m m m }
{
	\juliandate_calc:nnnn { #1 } { #2 } { #3 } { \use:n }
}
\NewDocumentCommand{\storejuliandate}{ s m m m m }
{
	\IfBooleanTF{#1}
	{
		\juliandate_calc:nnnn { #3 } { #4 } { #5 } { \cs_set:Npx #2 }
	}
	{
		\juliandate_calc:nnnn { #3 } { #4 } { #5 } { \cs_new:Npx #2 }
	}
}
\cs_new:Npn \juliandate_calc:nnnn #1 #2 #3 #4 % #1 = day, #2 = month, #3 = year, #4 = what to do
{
	#4 
	{
		\int_eval:n
		{
			#1 +
			\int_div_truncate:nn { 153 * (#2 + 12 * \int_div_truncate:nn { 14 - #2 } { 12 } - 3) + 2 } { 5 } +
			365 * (#3 + 4800 - \int_div_truncate:nn { 14 - #2 } { 12 } ) +
			\int_div_truncate:nn { #3 + 4800 - \int_div_truncate:nn { 14 - #2 } { 12 } } { 4 } -
			\int_div_truncate:nn { #3 + 4800 - \int_div_truncate:nn { 14 - #2 } { 12 } } { 100 } + 
			\int_div_truncate:nn { #3 + 4800 - \int_div_truncate:nn { 14 - #2 } { 12 } } { 400 } -
			32045
		}
	}
}

\tl_new:N \l__juliandate_g_tl
\tl_new:N \l__juliandate_dg_tl
\tl_new:N \l__juliandate_c_tl
\tl_new:N \l__juliandate_dc_tl
\tl_new:N \l__juliandate_b_tl
\tl_new:N \l__juliandate_db_tl
\tl_new:N \l__juliandate_a_tl
\tl_new:N \l__juliandate_da_tl
\tl_new:N \l__juliandate_y_tl
\tl_new:N \l__juliandate_m_tl
\tl_new:N \l__juliandate_d_tl
\int_new:N \l_juliandate_day_int
\int_new:N \l_juliandate_month_int
\int_new:N \l_juliandate_year_int

\cs_new:Npn \__juliandate_set:nn #1 #2
{
	\tl_set:cx { l__juliandate_#1_tl } { \int_eval:n { #2 } }
}
\cs_new:Npn \__juliandate_use:n #1
{
	\tl_use:c { l__juliandate_#1_tl }
}
\cs_new_protected:Npn \juliandate_reverse:n #1
{
	\__juliandate_set:nn { g }
	{ \int_div_truncate:nn { #1 + 32044 } { 146097 } }
	\__juliandate_set:nn { dg }
	{ \int_mod:nn { #1 + 32044 } { 146097 } }
	\__juliandate_set:nn { c }
	{ \int_div_truncate:nn { ( \int_div_truncate:nn { \__juliandate_use:n { dg } } { 36524 } + 1) * 3 } { 4 } }
	\__juliandate_set:nn { dc }
	{ \__juliandate_use:n { dg } - \__juliandate_use:n { c } * 36524 }
	\__juliandate_set:nn { b }
	{ \int_div_truncate:nn { \__juliandate_use:n { dc } } { 1461 } }
	\__juliandate_set:nn { db }
	{ \int_mod:nn { \__juliandate_use:n { dc } } { 1461 } }
	\__juliandate_set:nn { a }
	{ \int_div_truncate:nn { ( \int_div_truncate:nn { \__juliandate_use:n { db } } { 365 } + 1) * 3 } { 4 } }
	\__juliandate_set:nn { da }
	{ \__juliandate_use:n { db } - \__juliandate_use:n { a } * 365 }
	\__juliandate_set:nn { y }
	{
		\__juliandate_use:n { g } * 400 + 
		\__juliandate_use:n { c } * 100 + 
		\__juliandate_use:n { b } * 4 + 
		\__juliandate_use:n { a }
	}
	\__juliandate_set:nn { m }
	{ \int_div_truncate:nn { \__juliandate_use:n { da } * 5 + 308 } { 153 } - 2 }
	\__juliandate_set:nn { d }
	{ \__juliandate_use:n { da } - \int_div_truncate:nn { (\__juliandate_use:n { m } + 4) * 153 } { 5 } + 122 }
	\int_set:Nn \l_juliandate_year_int
	{ \__juliandate_use:n { y } - 4800 + \int_div_truncate:nn { \__juliandate_use:n { m } + 2 } { 12 } }
	\int_set:Nn \l_juliandate_month_int
	{ \int_mod:nn { \__juliandate_use:n { m } + 2 } { 12 } + 1 }
	\int_set:Nn \l_juliandate_day_int
	{ \__juliandate_use:n { d } + 1 }
}
\cs_generate_variant:Nn \juliandate_reverse:n { x }

\NewDocumentCommand{\showday}{ m }
{
	\juliandate_reverse:n { #1 }
	\int_to_arabic:n { \l_juliandate_day_int }-
	\int_to_arabic:n { \l_juliandate_month_int }-
	\int_to_arabic:n { \l_juliandate_year_int }
}

\NewDocumentCommand{\tomorrow}{ }
{
	\group_begin:
	\juliandate_reverse:x { \juliandate_calc:nnnn { \day + 1 } { \month } { \year } { \use:n } }
	\day = \l_juliandate_day_int
	\month = \l_juliandate_month_int
	\year = \l_juliandate_year_int
	\today
	\group_end:
}
\NewDocumentCommand{\tomorrowof}{ m m m }
{
	\group_begin:
	\juliandate_reverse:x { \juliandate_calc:nnnn { #1 + 1 } { #2 } { #3 } { \use:n } }
	\day = \l_juliandate_day_int
	\month = \l_juliandate_month_int
	\year = \l_juliandate_year_int
	\today
	\group_end:
}
\ExplSyntaxOff


\usepackage{xcolor,listings}
\usepackage{textcomp}
\begin{document} % конец преамбулы, начало документа
\input{data/title_05.tex}
\tableofcontents
\pagebreak

\section{Задание}

\begin{itemize}
	\item Подключиться к БД;
	\item Отразить схемы БД;
	\item Создать экранные формы с заменой вн. ключа;
	\item Наличие подчинённой экранной формы;
	\item 2 больших запроса, каждая таблица участвует как минимум 1 раз;
	\item 2 больших отчёта, каждая таблица участвует как минимум 1 раз;
	\item Поисковая форма;
	\item Главная форма;
\end{itemize}
\section{Выполнение работы}

\subsection{Схема БД}
	Были импортированы 4 таблицы базы данных -- клиенты, группы, расписание и тренера.
	Размещение связей представлен на \imref{img:5_schema}.
	
	\begin{figure}[H]
		\centering		
		\includegraphics[width=0.8\linewidth]{image/5_schema}
		\caption{Схема БД}\label{img:5_schema}
	\end{figure}

 
 \subsection{Экранные формы}
 
	 Далее представлены экранные формы.
	 В каждой из них внешние ключ заменены на читаемое описание (где имеются такие ключи).
	 На \imref{img:5_coach} и \imref{img:5_team} есть подтаблицы с описанием связанных данных.
	 На \imref{img:5_scheduler} и \imref{img:5_client} нет никаких подформ.
	 На \imref{img:5_team_client} дублирует форму для команд, но отображает связанных клиентов в отдельной форме.
	 
	 \begin{figure}[H]
	 	\centering		
	 	\includegraphics[width=\linewidth]{image/5_coach}
	 	\caption{Таблица тренеров с подтаблицей групп}\label{img:5_coach}
	 \end{figure}

	 \begin{figure}[H]
		\centering		
		\includegraphics[width=\linewidth]{image/5_scheduler}
		\caption{Таблица расписания}\label{img:5_scheduler}
	\end{figure}	 

	 \begin{figure}[H]
		\centering		
		\includegraphics[width=\linewidth]{image/5_team}
		\caption{Таблица команд с подтаблицей расписания}\label{img:5_team}
	\end{figure}
	
	\begin{figure}[H]
		\centering		
		\includegraphics[width=\linewidth]{image/5_client}
		\caption{Таблица клиентов}\label{img:5_client}
	\end{figure}	
	 \begin{figure}[H]
		\centering		
		\includegraphics[width=\linewidth]{image/5_team_client}
		\caption{Таблица команд со списком клиентов в них}\label{img:5_team_client}
	\end{figure}
	

 \subsection{Запросы}

	Данный запрос подсчитывает количество клиентов на каждый вид занятий группы в понедельник (\imref{img:5_req_group}).
	
	\begin{figure}[H]
		\centering		
		\includegraphics[width=0.4\linewidth]{image/5_req_group}
		\caption{Запрос список занятий по группам на понедельник}\label{img:5_req_group}
	\end{figure}
	
	
	\lstinputlisting[language=sql]{code/5_req_group.sql}
	
	
	Данный выводит расписание занятий в понедельник (\imref{img:5_req_scheduler}).
	
	\begin{figure}[H]
		\centering		
		\includegraphics[width=\linewidth]{image/5_req_scheduler}
		\caption{Запрос расписания на понедельник}\label{img:5_req_scheduler}
	\end{figure}
	
	
	\lstinputlisting[language=sql]{code/5_req_scheduler.sql}
	
	
	Данный выводит расписание каждого занятого тренера (\imref{img:5_req_coach_scheduler}).
	
	\begin{figure}[H]
		\centering		
		\includegraphics[width=\linewidth]{image/5_req_coach_scheduler}
		\caption{Запрос расписания тренеров}\label{img:5_req_coach_scheduler}
	\end{figure}
	
	
	\lstinputlisting[language=sql]{code/5_req_coach_scheduler.sql}
\subsection{Отчёты}

	Данный отчёт выводит список клиентов в каждой группе (\imref{img:5_report_client_team}).

	\begin{figure}[H]
		\centering		
		\includegraphics[width=\linewidth]{image/5_report_client_team}
		\caption{Отчёт по клиентам}\label{img:5_report_client_team}
	\end{figure}

	Данный отчёт нагрузку тренеров (\imref{img:5_report_coach_scheduler}).

	\begin{figure}[H]
		\centering		
		\includegraphics[width=\linewidth]{image/5_report_coach_scheduler}
		\caption{Отчёт по тренерам}\label{img:5_report_coach_scheduler}
	\end{figure}

\subsection{Поисковая форма}

	Поиск клиентов по ФИО (\imref{img:5_find_1}).
	
	\begin{figure}[H]
		\centering		
		\includegraphics[width=\linewidth]{image/5_find_1}
		\caption{Поиск клиентов по ФИО}\label{img:5_find_1}
	\end{figure}
	
	Поиск клиентов по ФИО и группе (\imref{img:5_find_2}).
	
	\begin{figure}[H]
		\centering		
		\includegraphics[width=\linewidth]{image/5_find_2}
		\caption{Поиск клиентов по ФИО и группе}\label{img:5_find_2}
	\end{figure}

\subsection{Итоговая форма}

	Итоговая форма со ссылками на основные элементы БД представлена на \imref{img:5_mainform}.
	
	\begin{figure}[H]
		\centering		
		\includegraphics[width=0.5\linewidth]{image/5_mainform}
		\caption{Главная форма}\label{img:5_mainform}
	\end{figure}
	
\section{Вывод}

В процессе выполнения практической работы получены навыки взаимодействия с СУБД MS Access и MySQL. 
В ходе выполнения работы приобретены знания о взаимодействии с пользовательским интерфейсом MS Access.
Были созданы формы отображения данных, в которых малоинформативное значения внешнего ключа заменялось удобочитаемым текстовым представлением \cite{sql2}.
Были созданы подчинённые формы для отображения связанных данных.
Были написаны запросы, в том числе с использованием соединения таблиц \cite{sql1}
Так же была произведена генерация удобных человекочитаемых отчётов о загруженности тренеров и групп.
Кроме того, была разработана поисковая форма для поиска клиентов с определённым именем или входящих в некоторую группу.
Была создана форма для удобной навигации между основными элементами решения.
Получены дополнительные навыки работы с несколькими таблицами, и сложными запросами.

\newpage 
\renewcommand{\refname}{{\normalsize СПИСОК ИСПОЛЬЗОВАННЫХ ИСТОЧНИКОВ}} 
\centering 
\begin{thebibliography}{2} 
	\addcontentsline{toc}{section}{\refname} 
	\bibitem{sql2} Кузин А. В., Демин В. М. Разработка баз данных в системе Microsoft Access. -- Издательство: "Форум", 2007. -- 223 с.
	\bibitem{sql1} Beaulieu A. Learning SQL: Master SQL Fundamentals. – " O'Reilly Media, Inc.", 2009.

	
\end{thebibliography}

\end{document} % конец документа